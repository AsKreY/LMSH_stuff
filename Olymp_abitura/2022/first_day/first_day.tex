\documentclass[a4paper, 12pt]{article}

\usepackage[russian, english]{babel}
\usepackage[utf8]{inputenc}
\usepackage[T2A]{fontenc}
\usepackage{geometry}
\usepackage{amsmath}
\usepackage{enumitem}

\geometry{a4paper, top=20mm, left=25mm, right=25mm}

\let\tan\relax
\DeclareMathOperator{\tan}{tg}
\DeclareMathOperator{\crg}{ctg}

\makeatletter
\AddEnumerateCounter{\asbuk}{\russian@alph}{щ}
\makeatother

\begin{document}
\clearpage
\pagestyle{empty}
\begin{enumerate}
    \item \textit{(Ломоносов)} Найдите сумму целых чисел $x\in[-11;5]$,
          удовлетворяющих неравенству
          \[\left(1-3\ctg^2\frac{\pi x}{12}\right)\left(1-\ctg^2\frac{\pi x}{12}\right)\left(1-\ctg\frac{\pi x}{4}\cdot\tg\frac{\pi x}{6}\right)\leq16\]
    \item \textit{(ОММО)} Вычислите
          \[\tg\frac{\pi}{43}\cdot\tg\frac{2\pi}{43}+\tg\frac{2\pi}{43}\cdot\tg\frac{3\pi}{43}+\cdots+\tg\frac{k\pi}{43}\cdot\tg\frac{(k+1)\pi}{43}+\cdots+\tg\frac{2019\pi}{43}\cdot\tg\frac{2020\pi}{43}\]
    \item \textit{(Ломоносов)} Решите неравенство
          \[\arcsin\left(\frac{5}{2\pi}\arccos x\right) > \arccos\left(\frac{10}{3\pi}\arcsin x\right)\]
    \item \textit{(Ломоносов)} Функция $y=f(t)$ такова, что сумма корней уравнения $f(\sin x)=0$ на отрезке $[\frac{3\pi}{2};2\pi]$ равна $33\pi$,а  а сумма корней уравнения $f(\cos x)=0$ на отрезке $[\pi;\frac{3\pi}{2}]$ равна $23\pi$. Какова сумма корней второго уравнения на отрезке $[\frac{\pi}{2};\pi]$?
    \item \textit{(Физтех)} Решите уравнение $(\cos x-3\cos 4x)^2=16+3\sin^23x$
\end{enumerate}
\end{document}