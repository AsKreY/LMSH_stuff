\documentclass[a4paper, 12pt]{article}

\usepackage[russian, english]{babel}
\usepackage[utf8]{inputenc}
\usepackage[T2A]{fontenc}
\usepackage{geometry}
\usepackage{amsmath}
\usepackage{enumitem}

\geometry{a4paper, top=20mm, left=25mm, right=25mm}

\let\tan\relax
\DeclareMathOperator{\tan}{tg}
\DeclareMathOperator{\crg}{ctg}

\makeatletter
\AddEnumerateCounter{\asbuk}{\russian@alph}{щ}
\makeatother

\begin{document}
\clearpage
\pagestyle{empty}
\begin{enumerate}
    \item \textit{Ломоносов} Две окружности касаются друг друга внутренним образом в точке $K$. Хорда $AB$ большей окружности касается меньшей окружности в точке $L$, причём $AL=10$. Найдите $BL$, если $AK:BK=2:5$.
    \item \textit{Физтех} Окружность $\Omega$ радиуса $\sqrt{3}$ касается сторон $BC$ и $AC$ треугольника $ABC$ в точках $K$ и $L$ соответственно и пересекает сторону $AB$ в точках $M$ и $N$ ($M$ лежит между $A$ и $N$) так, что отрезок $MK$ параллелен $AC, KC=1,AL=4$. Найдите $\angle ACB, MK, AB$ и площадь треугольника $CMN$.
    \item \textit{Ломоносов} Через вершину $A$ параллелограмма $ABCD$ проведена прямая, пересекающая диагональ $BD$, сторону $CD$ и прямую $BC$ в точках $E, F$ и $G$ соответственно. Найдите отношение $BE:ED$, если $FG:FE=$. Ответ при необходимости округлите до сотых.
    \item \textit{ОММО} В прямоугольном треугольнике $ABC$ на катете $AC$ как на диаметре построена окружность, которая пересекает гипотенузу $AB$ в точке $E$. Через точку $E$ проведена касательная к окружности, которая пересекает катет $CB$ в точке $D$. Найдите длину $DB$, если $AE=6$, а $BE=2$.
    \item \textit{Высшая проба !гроб!} Гипотенуза $AB$ прямоугольного треугольника $ABC$ касается вписанной и соответствующей вневписанной окружностей в точках $T_1, T_2$ соответственно. Окружность, проходящая через середины сторон, касается этих же окружностей в точках $S_1, S_2$ соответственно. Докажите, что $\angle S_1CT_1=\angle S_2CT_2$.
\end{enumerate}
\begin{enumerate}
    \item \textit{Ломоносов} Две окружности касаются друг друга внутренним образом в точке $K$. Хорда $AB$ большей окружности касается меньшей окружности в точке $L$, причём $AL=10$. Найдите $BL$, если $AK:BK=2:5$.
    \item \textit{Физтех} Окружность $\Omega$ радиуса $\sqrt{3}$ касается сторон $BC$ и $AC$ треугольника $ABC$ в точках $K$ и $L$ соответственно и пересекает сторону $AB$ в точках $M$ и $N$ ($M$ лежит между $A$ и $N$) так, что отрезок $MK$ параллелен $AC, KC=1,AL=4$. Найдите $\angle ACB, MK, AB$ и площадь треугольника $CMN$.
    \item \textit{Ломоносов} Через вершину $A$ параллелограмма $ABCD$ проведена прямая, пересекающая диагональ $BD$, сторону $CD$ и прямую $BC$ в точках $E, F$ и $G$ соответственно. Найдите отношение $BE:ED$, если $FG:FE=$. Ответ при необходимости округлите до сотых.
    \item \textit{ОММО} В прямоугольном треугольнике $ABC$ на катете $AC$ как на диаметре построена окружность, которая пересекает гипотенузу $AB$ в точке $E$. Через точку $E$ проведена касательная к окружности, которая пересекает катет $CB$ в точке $D$. Найдите длину $DB$, если $AE=6$, а $BE=2$.
    \item \textit{Высшая проба !гроб!} Гипотенуза $AB$ прямоугольного треугольника $ABC$ касается вписанной и соответствующей вневписанной окружностей в точках $T_1, T_2$ соответственно. Окружность, проходящая через середины сторон, касается этих же окружностей в точках $S_1, S_2$ соответственно. Докажите, что $\angle S_1CT_1=\angle S_2CT_2$.
\end{enumerate}
\end{document}