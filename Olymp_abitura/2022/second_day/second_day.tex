\documentclass[a4paper, 12pt]{article}

\usepackage[russian, english]{babel}
\usepackage[utf8]{inputenc}
\usepackage[T2A]{fontenc}
\usepackage{geometry}
\usepackage{amsmath}
\usepackage{enumitem}

\geometry{a4paper, top=20mm, left=25mm, right=25mm}

\let\tan\relax
\DeclareMathOperator{\tan}{tg}
\DeclareMathOperator{\crg}{ctg}

\makeatletter
\AddEnumerateCounter{\asbuk}{\russian@alph}{щ}
\makeatother

\begin{document}
\clearpage
\pagestyle{empty}
\begin{enumerate}
    \item \textit{Ломоносов} Арсению нравятся все числа, не делящиеся на $3$, а Тане нравятся числа, в которых нет цифр, делящихся на $3$.
          \begin{enumerate}[label=(\asbuk*)]
              \item Сколько четырёхзначных чисел нравятся и Арсению, и Тане?
              \item Найдите общую сумму всех таких четырёхзначных чисел.
          \end{enumerate}
    \item \textit{ОММО} Группа авантюристов показывает свою добычу. Известно, что ровно у $13$ авантюристов есть рубины; ровно у $9$ --- изумруды ; ровно у $15$ --- сапфиры; ровно у $6$ --- бриллианты. Кроме того, известно, что
          \begin{itemize}
              \item если у авантюриста есть сапфиры, то у него есть или изумруды, или бриллианты (но не то и другое одновременно);
              \item если у авантюриста есть изумруды, то у него есть или рубины, или сапфиры (но не то и другое одновременно).
          \end{itemize}
          Какое наименьшее количество авантюристов может быть в группе?
    \item \textit{ОММО} При каком наименьшем $n$ существуют $n$ чисел из интервала $(-1;1)$, таких, что их сумма равна $0$, а сумма их квадратов равна $30$?
    \item \textit{ММО} Волейбольный чемпионат с участием $16$ команд проходил в один круг (каждая команда играла с каждой ровно один раз, ничьих в волейболе не бывает). Оказалось, что какие-то две команды одержали одинаковое количество побед. Докадите, что найдутся три команды, которые выиграли друг у друга по кругу (то есть $A$ выиграла у $B$, $B$ выиграла у $C$, а $C$ выиграла у $A$).
    \item \textit{ММО} Султан собрал $300$ придворных мудрецов и предложил им испытание. Имеются колпаки $25$ различных цветов, заранее известных мудрецам. Султан сообщил, что
    на каждого из мудрецов наденут один из этих колпаков, причём если для каждого цвета
    написать количество надетых колпаков, то все числа будут различны. Каждый мудрец
    будет видеть колпаки остальных мудрецов, а свой колпак нет. Затем все мудрецы одновременно огласят предполагаемый цвет своего колпака. Могут ли мудрецы заранее
    договориться действовать так, чтобы гарантированно хотя бы $150$ из них назвали цвет
    верно?
    \item \textit{Высшая проба, !гроб!} Фокусник и его Ассистент готовятся показать следующий фокус. Фокуснику завяжут глаза, после чего один из зрителей напишет на доске $60$-битное слово (последовательность из $60$ нулей
    и единиц). Ассистент уверен, что сможет незаметно сообщить фокуснику $44$ бита (не обязательно написанные в слове, можно вычислять любые функции). После чего Фокусник должен
    будет назвать слово. Для какого наибольшего числа $C$ Фокусник и Ассистент могут придумать
    стратегию, позволяющую всегда назвать слово, совпавшее хотя бы в $C$ битах с написанным зрителем.
\end{enumerate}
\end{document}