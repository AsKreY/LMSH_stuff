\documentclass[a4paper, 11pt]{article}

\usepackage[russian, english]{babel}
\usepackage[utf8]{inputenc}
\usepackage[T2A]{fontenc}
\usepackage{geometry}
\usepackage{amsmath}
\usepackage{enumitem}

\geometry{a4paper, top=10mm, left=25mm, right=25mm}

\let\tan\relax
\DeclareMathOperator{\tan}{tg}
\DeclareMathOperator{\crg}{ctg}

\makeatletter
\AddEnumerateCounter{\asbuk}{\russian@alph}{щ}
\makeatother


\begin{document}
  \clearpage
  \pagestyle{empty}
  {\Large\bf Зачётный разнобой}
  \begin{enumerate}
    \item(\textit{ММО}) Известно, что сумма любых двух из трёх квадратных трёхчленов $x^2+ax+b$, $x^2+cx+d$, $x^2+ex+f$ не
    имеет корней. Может ли сумма всех этих трёхчленов иметь корни?
    \item(\textit{ММО})  Точки $O$ и $I$ --- центры описанной и вписанной окружностей неравнобедренного треугольника $ABC$. Две равные
    окружности касаются сторон $AB, BC$ и $AC, BC$ соответственно; кроме этого, они касаются друг друга в точке $K$. Оказалось, 
    что $K$ лежит на прямой $OI$. Найдите $\angle BAC$.
  \item(\textit{Ломоносов}) Из пункта $A$ в пункт $B$, расстояние между которыми равно 10 км, в 7:00 выехал автомобиль.
    Проехав $2/3$ пути, автомобиль миновал пункт $C$, из которого в этот момент в пункт
    $A$ выехал велосипедист. Как только автомобиль прибыл в $B$, оттуда в обратном направлении
    сразу же выехал автобус и прибыл в $A$ в 9:00. В скольких километрах от $B$ автобус догнал
    велосипедиста, если велосипедист прибыл в пункт $A$ в 10:00 и скорость каждого участника
    движения постоянна?
  \item(\textit{СПбГУ}) На однокруговой турнир по настольному теннису подало заявку 16 человек. Когда было
      сыграно $n$ матчей, окаазалось, что среди любых трёх теннисистов найдутся двое, уже сыгравших между
      собой. При каком наименьшем $n$ такое возможно?
    \item(\textit{Высшая проба}) Сколько точек, обе координаты которых натуральны, лежит строго внутри области,
      ограниченной осями координат и графиком функции \[-x^3+30x^2-300,6x+2012?\]
  \end{enumerate}
{\Large\bf Зачётный разнобой}
  \begin{enumerate}
    \item(\textit{ММО}) Известно, что сумма любых двух из трёх квадратных трёхчленов $x^2+ax+b$, $x^2+cx+d$, $x^2+ex+f$ не
    имеет корней. Может ли сумма всех этих трёхчленов иметь корни?
    \item(\textit{ММО})  Точки $O$ и $I$ --- центры описанной и вписанной окружностей неравнобедренного треугольника $ABC$. Две равные
    окружности касаются сторон $AB, BC$ и $AC, BC$ соответственно; кроме этого, они касаются друг друга в точке $K$. Оказалось, 
    что $K$ лежит на прямой $OI$. Найдите $\angle BAC$.
  \item(\textit{Ломоносов}) Из пункта $A$ в пункт $B$, расстояние между которыми равно 10 км, в 7:00 выехал автомобиль.
    Проехав $2/3$ пути, автомобиль миновал пункт $C$, из которого в этот момент в пункт
    $A$ выехал велосипедист. Как только автомобиль прибыл в $B$, оттуда в обратном направлении
    сразу же выехал автобус и прибыл в $A$ в 9:00. В скольких километрах от $B$ автобус догнал
    велосипедиста, если велосипедист прибыл в пункт $A$ в 10:00 и скорость каждого участника
    движения постоянна?
  \item(\textit{СПбГУ}) На однокруговой турнир по настольному теннису подало заявку 16 человек. Когда было
      сыграно $n$ матчей, окаазалось, что среди любых трёх теннисистов найдутся двое, уже сыгравших между
      собой. При каком наименьшем $n$ такое возможно?
    \item(\textit{Высшая проба}) Сколько точек, обе координаты которых натуральны, лежит строго внутри области,
      ограниченной осями координат и графиком функции \[-x^3+30x^2-300,6x+2012?\]
  \end{enumerate}
\end{document}
