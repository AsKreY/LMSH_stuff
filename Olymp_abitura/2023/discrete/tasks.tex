\documentclass[a4paper, 12pt]{article}

\usepackage[russian, english]{babel}
\usepackage[utf8]{inputenc}
\usepackage[T2A]{fontenc}
\usepackage{geometry}
\usepackage{amsmath}
\usepackage{enumitem}

\geometry{a4paper, top=20mm, left=25mm, right=25mm}

\let\tan\relax
\DeclareMathOperator{\tan}{tg}
\DeclareMathOperator{\crg}{ctg}

\makeatletter
\AddEnumerateCounter{\asbuk}{\russian@alph}{щ}
\makeatother

\begin{document}
  \clearpage
  \pagestyle{empty}
  {\Large\bf Дискретка}
  \begin{enumerate}
    \item Найдите все пары взаимно простых натуральных чисел $a$ и $b$, такие, что $a^2 + 2b^2$ делится на $a+2b$.
    \item В одной из вершин правильного $2n$-угольника, $n\ge2$, поставлено число $1$. Для данной расстановки чисел $2,\dots,2n$ в остальные 
    вершины $2n$-угольника поставим на каждой его стороне знак $+$, если число на конце стороны (при движении по часовой стрелке)
    больше числа на её начале и знак $-$, если оно меньше. Докажите, что модуль разности между числом расстановок чисел $2,\dots,2n$ с
    чётным числом плюсов на сторонах и числом расстановок с нечётным количеством плюсов равен числу расстановок, в которых плюсы и минусы
    чередуются при 
      \begin{enumerate}[label=(\asbuk*)]
        \item $n=3$
        \item $n=4$
        \item произвольном $n$.
      \end{enumerate}
    \item На доске написано произведение числе $\overline{\text{ИКС}}$ и $\overline{\text{КСИ}}$, где буквы соответствуют различным
    ненулевым десятичным цифрам. Это произведение шестизначное и оканчивается на С. Вася стёр с доски все нули, после чего там осталось
    ИКС. Что было написано на доске?
    \item Два числа $m$ и $n$ разлагаются на одинаковые простые множители, хотя различны; числа $m+1$ и $n+1$ также обладают этим свойством.
    Конечно или бесконечно множество пар $(m, n)$?
    \item Команда из $n$ школьников участвует в игре: на каждого из них надевают шапку одного из $k$ заранее известных
    цветов, а затем по свистку все школьники одновременно выбирают себе по одному шарфу. Команда получает столько
    очков, у скольких её участников цвет шапки совпал с цветом шарфа (шарфов и шапок любого цвета имеется достаточное
    количество; во время игры каждый участник не видит своей шапки, зато видит шапки всех остальных, но не имеет права
    выдавать до свистка никакую информацию). Какое наибольшее число очков команда, заранее наметив план действий
    каждого её члена, может гарантированно получить:
      \begin{enumerate}[label=(\asbuk*)]
        \item при $n=k=2$
        \item при произвольных фиксированных $n$ и $k$?
      \end{enumerate}
    \item  Докажите, что в таблице $8\times8$ нельзя расставить натуральные числа от $1$ до $64$ (каждое по одному разу) так,
      чтобы в ней для любого квадрата $2\times2$ вида \begin{tabular}{|c|c|}
      \hline
      a & b \\
      \hline
      c & d\\
      \hline
    \end{tabular} было выполнено равенство $|ad - bc| = 1$.
    \item День в Анчурии может быть либо ясным, когда весь
день солнце, либо дождливым, когда весь день льет дождь.
И если сегодня день не такой, как вчера, то анчурийцы го-
ворят, что сегодня погода изменилась. Однажды анчурий-
ские ученые установили, что 1 января день всегда ясный,
а каждый следующий день в январе будет ясным, только
если ровно год назад в этот день погода изменилась. В 2015
году январь в Анчурии был весьма разнообразным: то солн-
це, то дожди. В каком году погода в январе впервые будет
меняться ровно так же, как в январе 2015 года?
    \item Найдите наименьшее натуральное число $n$, для которого в десятичной записи $n$ вместе
      с $n^2$ используются все цифры от 1 до 9 {\bfровно} по одному разу.
  \end{enumerate}
\end{document}
