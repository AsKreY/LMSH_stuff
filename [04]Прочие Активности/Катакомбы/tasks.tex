\documentclass[a4paper, 12pt]{article}
\usepackage{amsmath, amsfonts}
\usepackage[english, russian]{babel}
\usepackage[T2A]{fontenc}
\usepackage[utf8]{inputenc}
\usepackage{geometry}
\geometry{a4paper, top=1mm, bottom=1mm, left=1mm, right=1cm}
\begin{document}
\begin{itemize}
  \item $S, F$ --- служебные символы, алфавит языка --- $\{a,b,c\}$, $S$ --- стартовый символ (изначально есть только он один), $\varepsilon$ --- пустое символ. Опишите языки, задаваемые следующими правилами:
  \begin{enumerate}
    \item $S\to FF, F\to FF, F\to ab$
    \item $S\to FS, S\to FF, F\to aFb, F\to\varepsilon$
    \item $F\to ab, F\to aFb, F\to FF$
    \item $S\to SaS\; |\; aSb\; |\; b$
  \end{enumerate}
  Мы можем заменять служебные символы в слове по правилам, слово считается принадлежащим языку, если его можно вывести из стартового символа по правилам.
  \item Дан шаблон $p$ и текст $t$. Найдите все вхождения $p$ в $t$ не более чем с одной ошибкой (то есть нужно найти подстроки $t$, которые равны $p$ с точностью до замены одного символа).
  \item Найдите количество способов, которыми можно замостить прямоугольник $M \times N$ симпатичными узорами, используя только плитки размера $1 \times 1$ чёрного и белого цвета. Узор считается симпатичным, если в нём нет квадрата $2 \times 2$ полностью покрытого плитками одного цвета.
\end{itemize}
\end{document}
