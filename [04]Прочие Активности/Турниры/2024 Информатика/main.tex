\documentclass[a4paper, 12pt]{article}

\documentclass[a4paper, 11pt]{article}

\usepackage[russian, english]{babel}
\usepackage[utf8]{inputenc}
\usepackage[T2A]{fontenc}
\usepackage{geometry}
\usepackage{amsmath}
\usepackage{enumitem}

\geometry{a4paper, top=10mm, left=25mm, right=25mm}

\let\tan\relax
\DeclareMathOperator{\tan}{tg}
\DeclareMathOperator{\crg}{ctg}

\makeatletter
\AddEnumerateCounter{\asbuk}{\russian@alph}{щ}
\makeatother


\title{Турнир по информатике}
\date{July 28, 2024}
\pagestyle{empty}


\begin{document}
    \maketitle
    \thispagestyle{empty}
    На данном турнире вам предлагается решить задачу по сжатию текста с помощью кода Хаффмана.

    Чтобы победить, вы должны реализовать самое эффективное сжатие --- разница между файлами до и
    после должна быть как можно больше при сохранении возможности получить исходный файл.

    Построение кода Хаффмана сводится к построению соответствующего бинарного дерева по следующему алгоритму:
    \begin{enumerate}
        \item Составим список кодируемых символов, при этом будем рассматривать один символ как дерево, состоящее из одного элемента c весом, равным частоте появления символа в строке.
        \item Из списка выберем два узла с наименьшим весом.
        \item Сформируем новый узел с весом, равным сумме весов выбранных узлов, и присоединим к нему два выбранных узла в качестве детей.
        \item Добавим к списку только что сформированный узел вместо двух объединенных узлов.
        \item Если в списке больше одного узла, то повторим пункты со второго по пятый.
    \end{enumerate}

    Реализцаия может быть на любом языке программирования, в результате должно получиться две программы: архиватор и деархиватор.
    В обоих случаях должен быть реализован запуск с использованием аргументов командной строки: \verb|./compress filename|, заархивированный файл должен называться "compressed",
    разархивированный --- "decompressed". Итоговые названия программ на ваше усмотрение.

    Вам надо сдать испольняемые файлы и сам код программ (в случае python'а это одно и то же).

    Часть текстов, на которых будут тестироваться ваши программы будут открытыми, мы выдадим их вам в начале.
\end{document}