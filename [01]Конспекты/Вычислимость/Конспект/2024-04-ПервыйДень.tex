Прежде чем разбирать любые алгоритмы и их результаты, для начала нам нужно понять, 
на чём и как эти алгоритмы исполняются и что вообще такое алгоритм.

\subsection*{Машина Тьюринга}

Кто такой Тюринг и зачем ему машина?

Всё просто --- это эдакий прапрапрадедушка компьютеров. Самая простая версия машины Тьюринга(МТ)
состоит из \textit{ленты} и \textit{автомата} --- набора состояний. Лента бесконечна
в обе стороны и состоит из полей, поначалу пустых (они обозначаются спецсимволом $\lambda$, который не обязательно
присутствует в алфавите). Также в МТ есть пишущая головка(ПГ), она
постоянно указывает на некоторое место на ленте и знает, какой символ там сейчас
находится. Символ этот принадлежит некоторому алфавиту, который также характеризует МТ.

Работа МТ заключается в выполнении действия, содержащегося в соответствующем состоянии и затем смене состояния
на основании символа под ПГ (пару символ-состояние называют \textit{конфигурацией МТ})

Теперь о состояниях. Каждое состояние характеризуется символом, который видит ПГ, символом который надо записать
в том месте, где ПГ находится, направлением, куда надо передвинуть ПГ (влево, вправо или оставить на месте) и
состоянием в которое надо перейти. Также есть два особых состояния: стартовое, его обычно обозначают $q_0$
и конечное --- $q!$, иногда ещё различаются два конечных состояния: $q_a$ (от слова accept) и $q_r$ (reject),
машина не обязательно приходит в конечное состояние, она вполне может зациклиться и не прекращать работу.

\textit{Попробуйте придумать пример подобной машины}

\textit{Подумайте, сколько состояний может быть у машины (вопрос с подвохом)}

Если вы уже знакомы с программированием, то наверное уже недоумеваете: "Как МТ взаимодействует с внешним миром?"
Даже в питоне изучение основ обычно начинается с изучения \verb*|input()|.
Всё довольно просто --- входом считается то, что записано на ленте изначально, то
есть, если лента не пуста, а существует некоторео стартовое слово (последовательность символов).
А выходом считается то, что находится на ленте после того, как машина остановилась (машина зациклилась --- отдельный результат).

Заметим, что на ленте необязательно должно находится входное слово, его запись может быть
"эмулирована" с помощью состояний МТ.

\begin{tcolorbox}[colback=blue!5,colframe=blue!75!black, title=Проблема останова]
    Забегая вперёд стоит отметить, что МТ тоже можно как-то закодировать и соответственно подать другой машине на вход
    на вход другой (или той же самой МТ). Это ставит целый ряд новых вопросов и вот один из довольно интересных:
    "Можем ли мы придумать машину, которая останавливается и возвращает 1, если поданная машина не остановится
    на поданном входе, а иначен зацикливается?". Попробуйте дома подумать над этой проблемой, а завтра
    мы с вами её обсудим
\end{tcolorbox}
\textit{Неразрешимость проблемы останова:} Докажем это от противного. Допустим, Анализатор существует. Напишем алгоритм Диагонализатор,
который принимает на вход число $N$, передает пару аргументов $(N,N)$ Анализатору
и возвращает результат его работы. Другими словами, Диагонализатор останавливается в
том и только том случае, если не останавливается алгоритм с номером $N$,
получив на вход число $N$. Пусть $K$ --- это порядковый номер Диагонализатора
в множестве $S$. Запустим Диагонализатор, передав ему это число $K$.
Диагонализатор остановится в том и только том случае, если алгоритм с номером
$K$ (то есть, он сам) не останавливается, получив на вход число $K$
(какое мы ему и передали). Из этого противоречия следует,
что наше предположение неверно: Анализатора не существует,
что и требовалось доказать. 

\subsection*{Практика}
На этом достаточно теории, пора поконтруировать эти самые МТ.
\begin{enumerate}
    \item $A=\{0,1,2,3,4,5,6,7,8,9\}$. Пусть $P$ --- непустое слово;
    значит, $P$ --- это последовательность из десятичных цифр,
    т.е.~запись неотрицательного целого числа в десятичной системе.
    Требуется получить на ленте запись числа, которое на $1$ больше числа $P$
    \item $A=\{a,b,c\}$. Перенести первый символ непустого слова $P$ в его конец
    \item $A=\{a,b,c\}$. Если первый и последний символы (непустого) слова $P$ одинаковы,
    тогда это слово не менять, а иначе заменить его на пустое слово.
    \item $A=\{a,b\}$. Удалить из слова $P$ его второй символ, если такой есть.
    \item $A=\{a,b,c\}$. Удалить из слова $P$ первое вхождение символа $a$, если такое есть.
    \item $A=\{a,b,c\}$. Удалить из $P$ все вхождения символа $a$.
    \item $A=\{a,b,c\}$. Удвоить слово $P$, поставив между ним и его копией знак $=$.
\end{enumerate}
\newpage
\subsection*{Домащнее задание}
\begin{enumerate}
    \item $A={a,b,c}$. Заменить на $a$ каждый второй символ в слове $P$.
    \item $A=\{a,b,c\}$. Определить, входит ли в слово $P$ символ $a$. Ответ: слово из
    одного символа $a$ (да, входит) или пустое слово (нет).
    \item $A=\{0,1,2,3\}$. Считая непустое слово $P$ записью числа в четверичной
    системе счисления, определить, является оно чётным числом или нет. Ответ: $1$
    (да) или $0$ (нет)
    \item $A=\{a,b,c\}$. Если $P$ – слово чётной длины $(0, 2, 4, \dots)$, то выдать ответ $a$,
    иначе --- пустое слово.
    \item $A=\{0,1,2\}$. Считая непустое слово $P$ записью числа в троичной системе
    счисления, определить, является оно чётным числом или нет. Ответ: $1$ (да) или $0$.
    \item $A=\{a,b,c\}$. Если слово $P$ имеет чётную длину, то оставить в нём только
    левую половину.
    \item $A=\{a,b\}$. Перевернуть слово $P$ (например: $abb\to bba$)
    \item $A=\{0,1\}$. Считая непустое слово $P$ записью двоичного числа, получить это
    же число, но в четверичной системе.
    \item $A=\{0,1,2,3\}$. Считая непустое слово $P$ записью числа в четверичной
    системе счисления, получить запись этого числа в двоичной системе.
    \item $A=\{ | \}$. Считая слово $P$ записью числа в единичной системе, определить,
    является ли это число степенью $3$ $(1, 3, 9, 27, \dots)$. Ответ: пустое слово, если
    является, или слово из одной палочки иначе.
    \item $A=\{ | \}$. Считая слово $P$ записью числа $n$ в единичной системе, получить в
    этой же системе число $2^n$.
    $A=\{(, )\}$. Определить, сбалансировано ли слово $P$ по круглым скобкам.
    Ответ: Д (да) или Н (нет).
    \item $A=\{a,b\}$. Если в $P$ символов $a$ больше, чем символов $b$, то выдать ответ $a$,
    если символов $a$ меньше символов $b$, то выдать ответ $b$, а иначе в качестве
    ответа выдать пустое слово.
\end{enumerate}
Невостребованные задачи встретятся с вами на зачёте.