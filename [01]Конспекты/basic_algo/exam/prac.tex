\documentclass[a4paper, 12pt]{article}

\usepackage[russian, english]{babel}
\usepackage[T2A]{fontenc}
\usepackage[utf8]{inputenc}

\begin{document}
  \section*{Практические задачи}
  \begin{enumerate}
    \item Герой по имени Магина сражается с группой из $n$ монстров с помощью легендарного топора, известного как Ярость Битвы. Каждый из монстров имеет $a_i$ очков здоровья. Каждым ударом топора Магина уменьшает здоровье того, кого он ударил, на $p$ очков, при этом уменьшая здоровье всех остальных монстров на $q$ очков. Монстр умирает, когда у него остается $0$ или менее очков здоровья. Магина хочет при каждом ударе выбирать цель таким образом, чтобы убить всех монстров за минимальное количество ударов. Требуется определить это количество.
    \item Напишите программу, которая для заданного массива $A=\langle a_1, a_2,\dots,a_n\rangle$ находит количество пар $(i,j)$ таких, что $i<j$ и $a_i>a_j$.
    \item Напишите структуру данных, которая за константу поддерживает извлечение минимума и максимума за константу, а также вставку за логарифм
    \item На числовой прямой окрасили $N$ отрезков. Известны координаты левого и правого концов каждого отрезка $[L_i,R_i]$. Найти сумму длин частей числовой прямой, окрашенных ровно в один слой. $N\le10000$. $L_i,R_i$ --- целые числа в диапазоне $[0,10^9]$.
    \item Компания Gnusmas разработала новую модель мобильного телефона. Основное достоинство этой модели --- ударопрочность: её корпус сделан из особого сплава, и телефон должен выдерживать падение с большой высоты. Компания Gnusmas арендовала $n$-этажное здание и наняла экспертов, чтобы те при помощи серии экспериментов выяснили, с какой высоты бросать телефон можно, а с какой --- нельзя. Один эксперимент заключается в том, чтобы бросить телефон с какого-то этажа и посмотреть, сломается он от этого или нет. Известно, что любой телефон этой модели ломается, если его сбросить с $x$-го этажа или выше, где $x$ --- некоторое целое число от $1$ до $n$, включительно, и не ломается, если сбросить его с более низкого этажа. Задача экспертов заключается в том, чтобы узнать число $x$ и передать его рекламному отделу компании. Задача осложняется тем, что экспертам предоставлено всего $k$ образцов новой модели телефона. Каждый телефон можно бросать сколько угодно раз, пока он не сломается; после этого использовать его для экспериментов больше не удастся. Подумав, эксперты решили действовать так, чтобы минимизировать максимально возможное количество экспериментов, которое может потребоваться произвести. Чему равно это количество?
    \item Группа математиков проводит бои между натуральными числами. Результаты боя между двумя натуральными числами, вообще говоря, случайны, однако подчиняются следующему правилу: если одно из чисел не менее чем в два раза превосходит другое, то большее число всегда побеждает; в противном случае победить может как одно, так и другое число. Бой называется неинтересным, если его результат предопределён. Множество натуральных чисел называется мирным, если бой любой пары различных чисел из этого множества неинтересен. Силой множества называется сумма чисел в нём. Сколько существует мирных множеств натуральных чисел силы $n$?
    \item Напишите программу, которая для двух вершин дерева определяет, является ли одна из них предком другой.
    \item Президент одной из новоиспеченных демократий на Ближнем Востоке после своего избрания взял курс на экономическое и социальное развитие страны. К сожалению, первая проблема, с которой столкнулся президент --- плачевное состояние дорожной системы в стране. В силу того, что дороги однонаправленные, далеко не всегда можно от одного города добраться до другого, а это мешает развитию коммуникаций между городами и экономики вообще. Вам поручено разработать план модернизации дорожной системы таким образом, чтобы от каждого города можно было добраться до каждого. План состоит в добавлении нескольких новых дорог. В силу того, что демократия в стране молодая, денег не хватает, поэтому количество добавленных дорог должно быть минимально возможным. Гарантируется, что для любых $i$ и $j$ из города $i$ в город $j$ ведёт не более одной дороги. Однако допустимы дороги из города в себя.
  \end{enumerate}
  \section*{Теоретические задачи}
  \begin{enumerate}
    \item Изначально есть массив $a_1 , a_2 , \dots , a_n$. К нему применяются $q$ преобразований вида $l, r, x$, что означает, что числа с $l$-го по $r$-е нужно увеличить на $x$. Выведите массив после всех преобразований. Асимптотика: $O(n + q)$.
    \item Пусть $a = (a_1, \dots , a_n ), b = (b_1 , \dots, b_m )$ --- две последовательности. Говорят, что $a$ является подпоследовательностью $b$, если из $b$ можно вычеркнуть некоторые элементы так, чтобы получилась $a$ (без изменения порядка оставшихся элементов). Формальнее, $a$ является подпоследовательностью $b$, если существует набор $1 \le i+1 < i_2 < \dots < i_n \le m$, такой что $b_{i_j} = a_j$ для всех $j \in \{1, \dots, n\}$. За $O(m)$ определите, является ли $a$ подпоследовательностью $b$. 
    \item Пусть $A$ --- массив длины $n$, а $B$ --- его отсортированная версия. Найдите за $O(n\log n)$ перестановку $\sigma$, такую что $b_i = a_{\sigma(i)}$ для всех $i$. В массиве $A$ могут быть повторяющиеся элементы.
    \item Пусть в алгоритме быстрой сортировки в качестве пивота всегда детерминированно выбирается центральный элемент массива. Для произвольного $n$ за $O(n)$ постройте перестановку, на котором такая сортировка занимает $\Omega(n^2)$ времени.
    \item На плоскости дано $n$ кругов. Любые два либо не пересекаются, либо вложены друг в друга. Найдите наибольшую цепочку вложенных кругов. Асимптотика: $O(n^2)$.
    \item Есть массив $a_1, \dots, a_n$ и два игрока, которые ходят по очереди. Каждым своим ходом игрок откусывает от массива крайнее левое или крайнее правое число. Результат игрока --- сумма всех откушенных им чисел. Каждый хочет максимизировать свой выигрыш. Кто выигрывает при правильной игре? Приведите алгоритм за $O(n^2)$.
    \item Докажите, что отношение существования простого пути является отношением эквивалентности.
    \item Деревом называется связный граф без циклов. Докажите, что дерево является двудольным графом.
    \item Найдите число путей в данном ориентированном ациклическом графе за $O(n + m)$.
  \end{enumerate}
\end{document}
