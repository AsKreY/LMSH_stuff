\documentclass[a4paper, 12pt]{article}

\usepackage[russian, english]{babel}
\usepackage[T2A]{fontenc}
\usepackage[utf8]{inputenc}
\usepackage{epigraph}
\usepackage{amsmath}

\title{Зачёт Сложности}

\begin{document}
    \maketitle
    \begin{enumerate}
        \item Перечеислите изученные на занятиях классы сложностей, расскажите про отношения между ними
        \item Рассмотрим двухленточные машины Тьюринга, которые на каждом шаге на каждой из лент либо
        и меняют символ, и сдвигают указатель, либо не меняют символа и остаются на месте (возможно, на
        разных лентах делают разное).
        \begin{enumerate}
            \item Дайте формальное определение машин с таким свойством как кортежей определённого вида.
            \item Докажите, что на машине такого вида можно смоделировать классическую одноленточную машину
            Тьюринга с не более чем полиномиальным замедлением.
        \end{enumerate}
        \item Пусть $A\in \mathbf{NP}$, при этом $A$ понимается как множество натуральных чисел. Докажите, что
        множество чисел $m$, таких что для некоторого $k\in A$ верно $m\vdots k^2$, также лежит в $\mathbf{NP}$.
        \item Определим класс $\mathbf{NP'}$ следующим образом: $A\in\mathbf{NP'}$ тогда и только тогда, когда существует
        $V(x, s)$, вычислимый за время $poly(|x|)$, со следующим условием:
\[x \in A \Leftrightarrow \exists s(V (x, s) = 1;s *) \]

$*s$ является кодом графа, в котором есть клика размером в половину графа

(Произвольная строка интерпретируется как код графа так: дополняется нулями до строки с длиной,
равной полному квадрату, полученная строка интерпретируется как матрица смежности. Возможные
петли, т.~е. единицы на диагонали, игнорируются. Размеры клики и графа считаются как число вершин).
Докажите, что $\mathbf{NP'}=\mathbf{NP}$. (Не забудьте доказать оба включения).
    \item Пусть $\textbf{ONLY-ODD-DEGREES} = \{k | \text{в разложение }k$ на простые множители все множители входят
    в нечётных степенях $\}$. Лежит ли этот язык в $\mathbf P, \mathbf{NP}, \mathbf{coNP}$? Докажите утверждения, которые можете
    доказать, а догадки – сформулируйте и поясните интуицию.
    \item Пусть $\mathbf{HAMPATHCYCLE} = \{(G, s, t) | $в ориентированном графе $G$ есть непересекающиеся путь и
    цикл, такие что $s$ является началом пути, $t$ --- его концом, а каждая вершина входит либо в путь, либо
    в цикл$\}$. Докажите, что этот язык является $\mathbf{NP}$-полным.
    \item Докажите, что язык $\mathbf{HALT} = \{n |$ машина Тьюринга с номером $n$ (в некоторой фиксированной
    нумерации) останавливается на входе $n\}$ является $\mathbf{NP}$-трудным, но не является $\mathbf{NP}$-полным.
    \end{enumerate}
\end{document}