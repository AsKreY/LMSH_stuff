\documentclass[a4paper, 12pt]{article}

\usepackage[english, russian]{babel}
\usepackage[utf8]{inputenc}
\usepackage[T2A]{fontenc}

\begin{document}
	\pagestyle{empty}
	\begin{enumerate}
		\item Что значит $f=O(g); f=\Omega(g);f=\Theta(g)$
		\item Сформулируйте мастер-теорему
		\item Пусть $T(n)=3T(\sqrt{n})  + \log_2n$. Найдите асимптотику $T(n)$
		\item Пусть $T(n)=2T(\frac{n}{2}) + n\log_2n$. Найдите асимптотику $T(n)$
		\item В массиве из нулей и единиц длины $n$ первый и последний элемент различный. За $O(\log n)$ найдите две соседние позиции в массиве, на которых стоят различные элементы
		\item Расскажите про устройство кучи и бинарной кучи
		\item Определите количество вершин на глубине $k$ в биномиальном дереве порядка $n$.
		\item За сколько работал бы в биномиальной куче \verb|SiftDown|
		\item* Пусть к изначально пустой биномиальной куче поступает $n$ запросов типа \verb|insert|. Докажите, что	она обрабатывает их за суммарное время $O(n)$, хотя некоторые запросы требуют $\Omega(\log n)$ операций.
		\item Число $0$ записано в $n$-разрядной двоичной системе. К нему $2n - 1$ раз прибавляется единица. Будем
		считать, что время, необходимое на прибавление единицы, равно количеству единиц в двоичной записи
		числа, которые становятся нулями. Оцените среднюю сложность всех таких операций. Какие операции
		являются самыми дешёвыми, а какие --- самыми дорогими?
	\end{enumerate}
\end{document}