\documentclass[a4paper, 12pt]{article}

\usepackage[english, russian]{babel}
\usepackage[T2A]{fontenc}
\usepackage[utf8]{inputenc}
\usepackage{verbatim}

\title{Конспект пары по Формальным языкам}
\author{Калмыков Андрей}


\begin{document}
\maketitle
\newpage
\section*{Первый день}
Основная цель математической логики --- формализация математических рассуждений. В определённый момент возникла идея сведениия любого объекта к последовательности некоторых символов. Не смотря на провал программы по сведению математики к чистому синтаксису, аппарат, позволяющий описывать некоторый язык с помощью строгого определения разрешённых последовательностей оказался полезен во многих других отраслях, например, компиляторы в программировании. Именно этот аппарат и является предметом данного курса.
\end{document}