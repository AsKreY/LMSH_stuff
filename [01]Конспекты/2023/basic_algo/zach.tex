\documentclass[a4paper, 12pt]{article}

\usepackage[russian, english]{babel}
\usepackage[T2A]{fontenc}
\usepackage[utf8]{inputenc}
\usepackage{epigraph}

\title{Зачёт Алгоритмы}
\date{}

\begin{document}
    \maketitle
    \epigraph{Фибоначиева куча лучше бинарной и биномиальной, но на проде не используется}{Неизвестный}
    Первые пять задач теоретиечские, вторые пять --- практика, и последняя с предпоследней --- "проектные"
    \begin{enumerate}
        \item Вам дан массив $a_1, a_2, \dots , a_n$, где $a_i$ --- цена на акции в следующие $n$ дней. Найдите два дня,
        в которые выгоднее всего купить и продать акции. Асимптотика: $O(n)$.
        \item Дан массив из $n$ чисел $a_1 , a_2 , \dots , a_n$ . Необходимо обработать $q$ запросов вида $_li, r_i , b_i , d_i$. В
        ответ на такой запрос нужно увеличить число $a_{l_i}$ на $b_i$, число $a_{l_i +1}$ увеличить на $b_i + d_i$ и так далее
        вплоть до $a_{r_i}$ , которое нужно увеличить на $b_i +d_i \cdot (r_i -l_i)$. Неформально, на подотрезке нужно прибавить
        арифметическую прогрессию. Выведите массив после всех запросов. Асимптотика: $O(n + q)$.
        \item В массиве из нулей и единиц длины $n$ первый и последний элемент различны. За $O(\log n)$
        найдите две соседние позиции в массиве, на которых стоят различные элементы.
        \item В этой задаче использовать разрешается не более одной кучи. Разработайте структуру
        данных $S$, которая бы позволяла обрабатывать любой запрос из нижеперечисленных за $O(\log n)$, где $n$ --- текущий размер структуры:
        \begin{itemize}
            \item \verb|insert x|: вставить целое число $x$ в $S$;
            \item \verb|getMin|: сообщить минимальное число в $S$;
            \item \verb|getMax|: сообщить максимальное число в $S$;
            \item \verb|extractMin|: удалить минимальное число из $S$.
        \end{itemize}
        \item Разработайте стек, который умеет прибавлять ко всем хранящимся значениям произвольную поправку $x$ за $O(1)$. Иными словами, нужно реализовать операцию увеличения всех чисел в стеке на $x$.
        \item Дан массив из $n$ цифр, найдите минимальную сумму чисел, составленных из цифр из этого массива 
        \item Даны $n$ нестрого возврастающих массивов $A_i$ и $m$ нестрого убывающих массивов $B_j$.Все массивы имеют одну и ту же длину $l$. Далее даны $q$ запросов вида $(i,j)$, ответ на запрос --- такое $k$, что $\max(A_{ik},B_{jk}$) минимален. Если таких $k$ несколько, можно вернуть любое.
        \item Дан массив, проверьте, является ли он бинарной кучей.
        \item Вы наблюдаете за бейсбольной игрой со странным правилами. Изначально у вас пустой счёт, вам даётся список операций, которые нужно последовательно применять:
        \begin{itemize}
            \item Целое число $x$: запишите счёт $x$
            \item \verb|+|: запишите новый счёт --- сумму предыдущих двух
            \item \verb|D|: запишите счёт --- удвоенный последний счёт
            \item \verb|C|: удалите последний записанный счёт
        \end{itemize}
        В конце выведите последний записанный счёт
        \item Дана страка из английских символов, часть которых в верхнем регистре.Хорошей строкой называется строка, которая не имеет двух смежнных символов $s_i$ и $s_{i+1}$, где:
        \begin{itemize}
            \item $0\leq i\leq |s| - 2$
            \item $s_i$  нижнем регистре и $s_{i+1}$ --- та же буква, но в верхнем регисте или наоборот
        \end{itemize}
        Чтобы сделать строку хорошей, выберите два подряд идущих символа, делающих её плохой, и удалите их. Продолжайте, пока строка не станет хорошей.
        Верните получившуюся хорошую строку.
        \item Реализуйте кучу
        \item Реализуйте HeapSort
    \end{enumerate}
\end{document}