\documentclass{article}
\usepackage[russian]{babel}
\usepackage[T2A]{fontenc}
\usepackage[utf8]{inputenc}
\usepackage{amsmath, amsfonts}
\title{Великий математический бой}
\date{}
\begin{document}
	\maketitle
	\pagestyle{empty}
	\begin{enumerate}
		\item Положительные числа $a, b, c$ удовлетворяют условию $abc(a+b+c) = 3$. Докажите, что $(a+b)(b+c)(c+a) \ge 8$ \\
		Решение. $(a+b)(b+c)(c+a) \ge  8 \Leftrightarrow (a+b)(b+c)(c+a)+abc = (a+b+c)(ab+bc+ca) \ge 8+abc \Leftrightarrow abc(a+b+c)(ab+bc+ca) = 3(ab+bc+ca) \ge 8abc+(abc)^2.$ Заметим, что$  3 = abc(a+b+c) \ge 3(abc)^{4/3},$ откуда $abc \le 1. Поэтому 8abc+(abc)^2 \le 9.$ С другой стороны$, (ab+bc+ca)^2 = a^2b^2+b^2c^2+a^2c^2+2abc(a+b+c). Заметим, что a^2b^2+b^2c^2+a^2c^2 = (a^2b^2+b^2c^2)/2+(b^2c^2+a^2c^2)/2+(a^2b^2+a^2c^2)/2 \ge ab^2c+bc^2a+ca^2b = abc(a+b+c). $Таким образом$, (ab+bc+ca)^2 \ge 3abc(a+b+c) = 9, $откуда$ 3(ab+bc+ca) \ge 3 \cdot 3 \ge 8abc+ (abc)^2 $что и завершает доказательство.
		\item У Влада и Саши есть $n \ge 5000$ бананов. Они по очереди съедают несколько бананов, причем каждый может съесть на один банан меньше или на один банан больше, чем перед этим съел другой (совсем ничего не есть нельзя). Первым ходит Влад (и может этим ходом съесть сколько угодно). Он хочет, чтобы после какого–то хода Бабуина осталось 2016 или 16 бананов. При каких $n$ Влад может победить?\\
		Ответ. При n = 3k и n = 3k+1. Решение. Пусть n при делении на 3 дает остатки 0 или 1. Первым ходом Павиан съедает один банан. Бабуин вынужден съесть 2 банана. Затем Павиан снова ест один банан и т.д. После каждой пары таких ходов количество бананов уменьшается на 3, и рано или поздно станет равным либо делящемуся на 3 числу 2016, либо числу 16, дающему при делении на 3 остаток 1. Пусть n при делении на 3 дает остаток 2. Тогда Бабуин всегда ест на один банан больше, чем Павиан на предыдущем ходу, пока не окажется, что очередной такой ход приводит к поражению. Если Павиан всё время ел по одному банану, такого не случится, потому что после хода Бабуина число бананов всегда будет давать при делении на 3 остаток 2. Значит, Павиан в какой-то момент съел больше одного банана, и дальше тоже всё время ел больше одного банана. Поэтому у Бабуина в момент опасности есть возможность съесть на один банан меньше, чем Павиан предыдущим ходом. Если после этого осталось 18 бананов, Бабуин с Павианом двумя следующими ходами вместе съедят не меньше 3 бананов, и Бабуин не проиграет. Если же осталось 2018 бананов, то тактика 1-2-1-2-… приведет Павиана к поражению, потому что 16 не дает при делении на 3 остатка 2, а если Павиан каким-то ходом съест больше одного банана, Бабуин не даст ему выиграть уже описанным выше способом.
		\item Пусть $A = 11 \dots 11 (1526 единиц)$. При каком наибольшем $n$ не существует натурального числа $B$, кратного $A$, сумма цифр которого равна $n$?
		\item Решите в натуральных числах уравнение $n^2m^5-2^n5^m = 30+4nm.$
		\item На выездную олимпиаду в ЛМШ приехали 2023 ученика. Согласно новому Порядку, если на 1-ю неделю приехали $a$ учеников, на 2-й осталось $b$ учеников, а на 3-й --- c учеников, то $a-b$ должно равняться $b-c$. Поскольку ученик очередной недели должен быть учеником и предыдущего, a $\ge$ b $\ge$ c. Сколькими способами дирекция ЛМШ может выбрать количество детей на очередной неделе? Варианты считаются различными, если они отличаются составом учеников хотя бы на одной неделе.\\
		Ответ. C_{2023}^{4046}. Решение. Если перевести на язык множеств, то нам надо выбрать три вложенных друг в друга подмножества C  B  A  \{1,2,3,...,2016\}, в которых a, b и c элементов соответственно так, что b–a = c–b. Рассмотрим множество из 4032 элементов x1, x2, ..., x2016, y1, y2, ..., y2016 и какой-нибудь способ выбрать из него 2016 элементов. Сопоставим ему способ выбрать тройку подмножеств с заданным условием. Будем считать, что элемент k лежит в C, если xk выбрано, а yk нет; лежит в B, если выбрано xk; лежит в A, если выбрано xk или не выбрано yk. Легко видеть, что тогда b–a = c–b. Таким образом, построенное соответствие дает приведенный выше ответ.
		\item Дан остроугольный разносторонний треугольник. С помощью циркуля и линейки проведите две прямые, делящие данный треугольник на четыре части так, что из них можно сложить прямоугольник, и ни одна из прямых не параллельна стороне треугольника.
	\end{enumerate}
\end{document}