\documentclass{article}
\usepackage[russian]{babel}
\usepackage[T2A]{fontenc}
\usepackage[utf8]{inputenc}
\usepackage{geometry}
\usepackage{amsmath, amsfonts}
\geometry{
	a4paper,
	top=1cm,
	bottom=1cm}


\begin{document}
\clearpage
\pagestyle{empty}
\begin{center}
	\bf Памятка по мировому господству
\end{center}
\textbf{Ход игры}
Существует $7$ стран. В каждой по $3$ человека. Страны формируются сами и сами выбираю презедента. Только президент решает куда тратить деньги страны.

Игра проходит в $6$ раундов. Каждый раунд поделен на фазы:
\begin{enumerate}
	\item \textit{Переговоры}. Президент каждой страны решает, на что потратить деньги в этом раунде. ($12$ минут). В первые две минуты возможно объявление от президента
	\item \textit{Дипломатия}. ГМ предлагает на рассмотрение $2$ закона. Все страны голосуют за один из законов, при равенстве голосов принимаются оба закона. Законы действуют на всех. Уничтоженные страны имеют $2$ голоса вместо $1$.
	\item \textit{Подсчет}. Активируются действия всех стран, перерасчитывается общедоступная инфа. ($2$ минуты)
\end{enumerate}
%$12$-$13$ минут на переговоры, во время которых президенты говорят организатору куда тратить деньги. После этих $12$ минут наступает фаза, в которой все действия активируются и пересчитываются и стираются различные общедоступные параметры, данная фаза занимает пару минут.
После 6 раунда объявляется победитель.\\
\textbf{Параметры стран}
\begin{itemize}
\item Монеты. В начале у каждой страны по $2000$~монет. 
%\item Города. Города приносят деньги в зависимости от уровня ($400$, $500$,  $600$, $700$~монет в ход). Прокачка города стоит $300$ ~монет На город можно купить щит ($700$~монет), который сразу будет установлен. На один город можно установить максимум $1$~щит. Щит защищает от одной ракеты, но может быть разрушен. Если город взорван ракетой, он не приносит денег. Если у страны разбомблены $4$~города, она выбывает из игры. %What did u mean by last???? Question of price
\item Города. Города приносят деньги в зависимости от уровня (Каждый $300+100 \cdot \text{уровень} $ монет в раунд). Изначально у каждой страны 4 города.
\item Завод. Покупается один раз, позволяет строить ракеты. Изначально заводов у стран нет. %price
\item Ракеты. После постройки завода можно покупать  неограниченно ракет за, $300$~монет штуку. Изначально ракет у стран нет.%В один город одна страна может отправить максимум $1$~ракету. %price
\item Экология. В конце игры выигрывает выжившая страна, имеющая наибольшую экологию. Экология изначально у всех $100\%$. %После покупки завода любой страной у \textbf{ВСЕХ} стран экология снижается на $10\%$. Можно купить $10\%$~экологии и это стоит $300$~монет. %price, формулировка по городам
\end{itemize}
В конце раунда каждая страна получает монеты по формуле $$\text{монеты} = \frac{\text{производство} \cdot \text{экология}}{100}$$%strange formula
\\
\textbf{Переговоры}
В фазу переговоров президент страны может решить, какие из перечисленных действий осуществить:
\begin{itemize}
\item Купить завод. Покупается один раз за игру, позволяет производить ракеты. После покупки экология всех стран падает на $10\%$. Cтоит 1000 монет.
\item Производить ракеты.  Стоит 300 монет за штуку. 
\item Отправление ракет. На один город можно отправить только одну ракету. Если у города не было щита, он будет разбомблен, иначе город потеряет щит.
\item Производить щиты на город. Щит защищает город от одной ракеты. Можно купить только 1 щит за раунд. Стоит 350 монет.
\item Покупать экологию. За каждые 300 монет можно купить $10\%$ экологии.
\item Прокачать город. Любой город можно прокачать вплоть до 4 уровня. За каждый уровень производство города увеличивается на $100$ монет. Стооит 300 монет.
\end{itemize}
\textbf{Что видят все страны}\\
Всем странам доступная общая таблица, отражающая уровни городов и экологии каждой из стран. Таблица обновляется после каждого раунда. %Показывается ли разрушенность города?
\\
%\textbf{Дипломатия}
%В конце хода перед фазой подсчёта ГМ предлагает на рассмотрение $2$ закона. Все страны голосуют за один из законов, при равенстве голосов принимаются оба закона. Законы действуют на всех. Уничтоженные страны имеют $2$ голоса вместо $1$.
\\
\textbf{Краткая сводка цен}\\
\\
\begin{tabular}{|c|c|c|c|c|c|}
	\hline
	\textbf{Предмет} & Щит & Завод & Ракета & $10\%$ экологии & Улучшение города \\
	\hline
	\textbf{Цена} & $350$ & $1000$ & $300$ & $300$ & $300$\\
	\hline
	
\end{tabular}
\end{document}
