\documentclass[a4paper, 12pt]{article}

\usepackage[russian, english]{babel}
\usepackage[T2A]{fontenc}
\usepackage[utf8]{inputenc}
\usepackage{epigraph}
\usepackage{amsmath}

\title{Зачёт Алгоритмы}

\begin{document}
\maketitle
    \begin{enumerate}
        \item Разработайте структуру данных $S$, которая бы позволяла хранить множество целых чисел,
        добавлять в него элементы и удалять $|S| / 2$ наибольших элементов из множества. Асимптотика: $O(1)$
        амортизированно (то есть $q$ последовательных запросов к изначально пустому $S$ должны обрабатываться за $O(q)$).
        \item  Задан массив $a(0),\dots,a(2^n-1)$. Определим $a'(mask)=\sum\limits_{submask\subseteq mask} a(submask)$. Докажите, что
        следующий код решает эту задачу на месте (результат сохраняется в исходном массиве):
        \begin{verbatim}
void magic(vector<int>& a) {
    for (int i = 0; i < n; ++i)
        for (int mask = 0; mask < (1 << n); ++mask)
            if (!bit(mask, i))
                a[mask + (1 << i)] += a[mask];
}
        \end{verbatim}
        \item  На гранях шестигранного кубика могут располагаться числа от $1$ до $n$, повторы не запрещены. Два
        кубика считаются различными, если на кубиках различны мультимножества расположенных чисел.
        Скажем, что один кубик превосходит другой, если с вероятностью, строго большей $\frac12$, при случайном
        равномерном бросании обоих кубиков на первом выпадает большее число. Назовём тройку кубиков
        хорошей, если первый кубик превосходит второй, второй превосходит третий, а третий превосходит
        первый. Определите число хороших упорядоченных троек кубиков за:
        \begin{enumerate}
            \item $O(n)$
            \item $O(\log n)$
            \item $O(1)$
        \end{enumerate}
        \item Дан связный неориентированный граф $G$. Нужно ориентировать как можно больше его
        рёбер, так чтобы по-прежнему из каждой вершины был путь в каждую. Асимптотика: $O(n + m)$.
        \item Пусть $\phi$ --- формула в виде $2$-КНФ с $n$ переменными и $m$ скобками. За $O(n \cdot (n + m))$
        определите для каждой переменной верно ли, что её значение одинаково во всех выполняющих наборах
        $\phi$. Иными словами, обязательно ли значение переменной фиксировано, если $\phi=1$?
        \item Продавец аквариумов для кошек хочет объехать $n$ городов, посетив каждый из них ровно один раз. Помогите ему найти кратчайший путь. (На вход даётся матрица смежности, на выход выдаётся длина и список городов)
        \item Группа математиков проводит бои между натуральными числами. Результаты боя между двумя натуральными числами, вообще говоря, случайны, однако подчиняются следующему правилу: если одно из чисел не менее чем в два раза превосходит другое, то большее число всегда побеждает; в противном случае победить может как одно, так и другое число.
        Бой называется неинтересным, если его результат предопределён. Множество натуральных чисел называется мирным, если бой любой пары различных чисел из этого множества неинтересен. Силой множества называется сумма чисел в нём. Сколько существует мирных множеств натуральных чисел силы $n$?
        \item В компьютерной сети вашей фирмы $n$ компьютеров. В последнее время свитч, к которому они подключены, сильно барахлит,
        и потому не любые два компьютера могут связаться друг с другом. Кроме того, если компьютер $a$ обменивается информацией с компьютером $b$,
        то никакие другие компьютеры не могут в это время обмениваться информацией ни с $a$, ни c $b$. Вам необходимо вычислить максимальное количество компьютеров,
        которые могут одновременно участвовать в процессе обмена информацией. (На вход дана матрциа смежности)
        \item Есть $n$ лампочек и $m$ переключателей. Каждый переключатель контролирует некоторое множество лампочек, а вот каждая лампочка контролируется ровно
        двумя переключателями. Если изменить положение переключателя, то изменят своё состояние все контролируемые им лампочки: горящие потухнут, а негорящие
        зажгутся. Определите, можно ли так нажать на некоторые (возможно, никакие) переключатели, чтобы все лампочки зажглись.

        \textbf{Входные данные}

        Первая строка содержит два целых числа $n$ и $m$ ($2\leq n\leq105, 2\leq m\leq105$) --- число лампочек и число переключателей.

        Следующая строка содержит $n$ целых чисел $r_1, r_2,\dots,r_n(0\leq r_i\leq 1)$ --- изначальные состояния лапочек. Лампочка $i$ включена, если и только если $r_i=1$.

        В каждой из следующих $m$ строк содержится целое число $x (0\leq x \leq n)$, а затем $x$ различных целых чисел --- количество лампочек, контролируемых очередным переключателем, а затем номера этих лампочек. Гарантируется, что каждая лампочка контролируется ровно двумя переключателями.
    \end{enumerate}
\end{document}