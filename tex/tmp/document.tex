\documentclass[a4paper, 12pt]{article}
\usepackage[english, russian]{babel}
\usepackage[utf8]{inputenc}
\usepackage[T2A]{fontenc}

\usepackage{fancyhdr}
\pagestyle{fancy}
\rhead{1июля 2017г.}
\lhead{Ивлев Ф. А.}
\chead{группа 10-1}

\begin{document}
\begin{center}
	\Large{\textbf{Полярные преобразования}}
\end{center}

{\bfseries{Определение.}}
{\itshape{Полярой}}
точки
{\itshape{P}}
относительно окружности
$\omega$
называется прямая, проходящая через точку
{\itshape{Р'}}
инверсную точке
{\itshape{P}}
относительно окружности, и перпендикулярная прямой
{\itshape{PP'}}
{\bfseries{Определение.}}
{\itshape{Полюсом}}
прямой
{\itshape{l}}
относительно окружности
$\omega$
называется точка, являющаяся инверсным образом относительно этой окружности основания перпендикуляра, опущенного из центра
$\omega$
на
{\itshape{l}}
.
{\bfseries{Определение.}}
{\itshape{Полярным}}
преобразованием относительно окружности
$\omega$
называется преобразование, которое ставит каждой точке в соответствие её поляру, а каждой прямой в соответствие её полюс.

Основное свойство полярного преобразования (
{\itshape{полярная двойственность}}
): если полюс прямой
{\itshape{l}}
лежит на прямой
{\itshape{k}}
, то полюс прямой
{\itshape{k}}
лежит на прямой 
{\itshape{l}}
. И наоборот. Если поляра точки
{\itshape{A}}
проходит через
{\itshape{В,}}
то поляра точки
{\itshape{B}}
проходит через
{\itshape{A.}}

{\itshape{Упражнение.}}
Докажите, что три точки лежат на одной прямой тогда и только тогда, когда их поляры проходят через одну точку.
\begin{enumerate}
	\item{Треугольник}
{\itshape{ABC}}
вписан в окружность. Касательные к каждой его вершине пересекают продолжения противоположных сторон в точках
{\itshape{$A_1, B_1, C_1/$}}Доказать, что эти точки лежат на одной прямой.

\item{}
{\itshape{a}})
Представим, что мы доказали теорему Паскаля. Докажите теорему Брианшона.
b) Представим, что мы доказали теорему Дезарга в одну сторону. Докажите её теперь в другую сторону. с) А что если применить полярное преобразование к теореме Паппа?
\item{Дан полукруг}
{\itshape{S}}
с центром
{\itshape{O}}
и диаметром
{\itshape{АВ.}}
На
{\itshape{AB}}
выбрана точка произвольная точка
{\itshape{P}}.
Пусть
{\itshape{M}}
и
{\itshape{N}}
такие точки полукруга
{\itshape{S}}
, что 
$\angle$
{\itshape{APM=}}
$\angle$
{\itshape{BPN=}}
$\alpha$.
Докажите, что точки пересечения прямых
{\itshape{MN}}
и
{\itshape{AB}}
не зависят от
$\alpha$
\end{enumerate}
\end{document}