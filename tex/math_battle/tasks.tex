\documentclass{article}
\usepackage[russian]{babel}
\usepackage[T2A]{fontenc}
\usepackage[utf8]{inputenc}
\usepackage{amsmath, amsfonts}
\title{Великий математический бой}
\date{}
\begin{document}
	\maketitle
	\pagestyle{empty}
	\begin{enumerate}
		\item Положительные числа $a, b, c$ удовлетворяют условию $abc(a+b+c) = 3$. Докажите, что $(a+b)(b+c)(c+a) \ge 8$
		\item У Влада и Саши есть $n \ge 5000$ бананов. Они по очереди съедают несколько бананов, причем каждый может съесть на один банан меньше или на один банан больше, чем перед этим съел другой (совсем ничего не есть нельзя). Первым ходит Влад (и может этим ходом съесть сколько угодно). Он хочет, чтобы после какого–то хода Бабуина осталось 2016 или 16 бананов. При каких $n$ Влад может победить?
		\item Пусть $A = 11 \dots 11 (1526 единиц)$. При каком наибольшем $n$ не существует натурального числа $B$, кратного $A$, сумма цифр которого равна $n$?
		\item Решите в натуральных числах уравнение $n^2m^5-2^n5^m = 30+4nm.$
		\item На выездную олимпиаду в ЛМШ приехали 2023 ученика. Согласно новому Порядку, если на 1-ю неделю приехали $a$ учеников, на 2-й осталось $b$ учеников, а на 3-й --- c учеников, то $a-b$ должно равняться $b-c$. Поскольку ученик очередной недели должен быть учеником и предыдущего, a $\ge$ b $\ge$ c. Сколькими способами дирекция ЛМШ может выбрать количество детей на очередной неделе? Варианты считаются различными, если они отличаются составом учеников хотя бы на одной неделе.
		\item Дан остроугольный разносторонний треугольник. С помощью циркуля и линейки проведите две прямые, делящие данный треугольник на четыре части так, что из них можно сложить прямоугольник, и ни одна из прямых не параллельна стороне треугольника.
	\end{enumerate}
\end{document}