\documentclass[a4paper, 12pt]{article}

\usepackage[russian, english]{babel}
\usepackage[utf8]{inputenc}
\usepackage[T2A]{fontenc}
\usepackage{geometry}
\usepackage{amsmath}
\usepackage{enumitem}

\geometry{a4paper, top=20mm, left=25mm, right=25mm}

\let\tan\relax
\DeclareMathOperator{\tan}{tg}
\DeclareMathOperator{\crg}{ctg}

\makeatletter
\AddEnumerateCounter{\asbuk}{\russian@alph}{щ}
\makeatother

\begin{document}
  \clearpage
  \pagestyle{empty}
  {\Large\bf Алгебра}
  \begin{enumerate}
    \item Решите неравенство \[8\cdot\frac{|x+1|-|x-7|}{|2x-3|-|2x-9|}+3\cdot\frac{|x+1|+|x-7|}{|2x-3|+|2x-9|}\le8\]
    \item Найдите все значения $p$, при каждом из которых
    числа $4p+5, 2p$ и $|p-3|$ являются соответственно
    первым, вторым и третьим членами некоторой геометрической прогрессии.
    \item \phantom{sdaasdsadsad}   
    \item Решите уравнение \[\sqrt{\sqrt{x+2}+\sqrt{x-2}}=2\sqrt{\sqrt{x+2}-\sqrt{x-2}}+\sqrt{2}\]
    \item Докажите, что все положительные корни многочлена $x(x+1)(x+2)(x+3)(x+4)-2$ больше $\frac{1}{14}$.
    \item Когда к квадратному трёхчлену прибавили $x^2$, его наименьшее значение
    увеличилось на 1, а когда из него вычли $x^2$, его наименьшее значение уменьшилось на 3.
    А как изменится наименьшее значение $f(x)$, если к нему прибавить $2x^2$?
    \item Какое наибольшее значение может принимать выражение \[1:(a+2010:(b+1:c)),\]
    где $a,b,c$ --- попарно различные ненулевые цифры?
  \end{enumerate}
\end{document}
