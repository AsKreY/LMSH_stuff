\documentclass[a4paper, 12pt]{article}

\usepackage[russian, english]{babel}
\usepackage[utf8]{inputenc}
\usepackage[T2A]{fontenc}
\usepackage{geometry}
\usepackage{amsmath}
\usepackage{enumitem}

\geometry{a4paper, top=20mm, left=25mm, right=25mm}

\let\tan\relax
\DeclareMathOperator{\tan}{tg}
\DeclareMathOperator{\crg}{ctg}

\makeatletter
\AddEnumerateCounter{\asbuk}{\russian@alph}{щ}
\makeatother

\begin{document}
\clearpage
\pagestyle{empty}
\begin{enumerate}
    \item \textit{ММО ($2$ балла)} В коллекции Игоря есть два типа предметов: значки и браслеты. Значков больше, чем браслетов. Алик заметил, что если он увеличит количество браслетов в некоторое (не обязательно целое) чтсло раз, не изменив количества значков, то в его коллекции будет $100$ предметов. А если, наоборот, он увеличит в это же число раз первоначальное количество значков, оставив прежним количество браслетов, то у него будет $101$ предмет. Сколько значков и сколько браслетов могло быть в коллекции Алика?
    \item \textit{Ломоносов ($2$ балла)} В треугольнике $ABC$, площадь которого равна $20$, проведена медиана $CD$. Найдите радиус окружности, описанной около треугольника $ABC$, если известно, что $AC=\sqrt{41}$, а центр окружности, вписанной в треугольник $ACD$, лежит на окружности, описанной около треугольника $BCD$.
    \item \textit{ОММО ($2$ балла)}Бригада рабочих тудилась на заливке катка на большом и малом полях, причём площадь большого поля была в $2$ раза больше площади малого поля. В той части бригады, которая работлаа на большом поле, было на $4$ рабочих больше, чем в той части, которая работала на малом поле. Когда заливка большого катка закончилась, часть бригады, которая была на малом поле, ещё работала. Какое наибольшее число рабочих могло быть в бригаде?
    \item \textit{Физтех ($2$ балла)} Решите уравнение \[\frac{\cos x\cdot\cos \left(x+\dfrac{\pi}{4}\right)}{7\cos^2x+5\sin^2x-6}+\frac{\cos x\cdot\sin\left(x+\dfrac{\pi}{4}\right)}{6-5\cos^2x-7\sin^2x}=\frac{\tg 2x}{\sqrt{2}}\]
\end{enumerate}
\begin{enumerate}
    \item \textit{ММО ($2$ балла)} В коллекции Игоря есть два типа предметов: значки и браслеты. Значков больше, чем браслетов. Алик заметил, что если он увеличит количество браслетов в некоторое (не обязательно целое) чтсло раз, не изменив количества значков, то в его коллекции будет $100$ предметов. А если, наоборот, он увеличит в это же число раз первоначальное количество значков, оставив прежним количество браслетов, то у него будет $101$ предмет. Сколько значков и сколько браслетов могло быть в коллекции Алика?
    \item \textit{Ломоносов ($2$ балла)} В треугольнике $ABC$, площадь которого равна $20$, проведена медиана $CD$. Найдите радиус окружности, описанной около треугольника $ABC$, если известно, что $AC=\sqrt{41}$, а центр окружности, вписанной в треугольник $ACD$, лежит на окружности, описанной около треугольника $BCD$.
    \item \textit{ОММО ($2$ балла)}Бригада рабочих тудилась на заливке катка на большом и малом полях, причём площадь большого поля была в $2$ раза больше площади малого поля. В той части бригады, которая работлаа на большом поле, было на $4$ рабочих больше, чем в той части, которая работала на малом поле. Когда заливка большого катка закончилась, часть бригады, которая была на малом поле, ещё работала. Какое наибольшее число рабочих могло быть в бригаде?
    \item \textit{Физтех ($2$ балла)} Решите уравнение \[\frac{\cos x\cdot\cos \left(x+\dfrac{\pi}{4}\right)}{7\cos^2x+5\sin^2x-6}+\frac{\cos x\cdot\sin\left(x+\dfrac{\pi}{4}\right)}{6-5\cos^2x-7\sin^2x}=\frac{\tg 2x}{\sqrt{2}}\]
\end{enumerate}
\end{document}